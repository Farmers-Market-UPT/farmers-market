\documentclass[a4paper,11pt]{article}  

% ------------------ ENCODING & LANGUAGE ------------------
\usepackage[utf8]{inputenc} % UTF-8 encoding
\usepackage[T1]{fontenc}    % Proper font encoding for special characters
\usepackage[portuguese]{babel} % Portuguese hyphenation & translations

% ------------------ PAGE LAYOUT ------------------
\usepackage[left=2cm, right=2cm, top=2cm, bottom=2cm]{geometry} % Custom margins
\usepackage{setspace}   % Control line spacing
\onehalfspacing         % 1.5 line spacing for better readability
\usepackage{parskip}    % Adds space between paragraphs instead of indentation

% ------------------ MATH & SYMBOLS ------------------
\usepackage{amsmath, amssymb, amsthm} % Core math packages
\usepackage{mathtools} % Extends amsmath (e.g., better equation numbering)
\usepackage{bm} % Bold symbols in math mode
\usepackage{siunitx} % SI units, e.g., \SI{9.8}{m/s^2}

% ------------------ GRAPHICS & FIGURES ------------------
\usepackage{graphicx} % Include images
\usepackage{float} % Allow pictures to stay where I put them
\usepackage{xcolor} % Colors for text and diagrams
\usepackage{tikz} % Drawing graphs and diagrams
\usepackage{tikz}
\usetikzlibrary{decorations.pathreplacing}
\usepackage{pgfplots} % Plot functions and data
\pgfplotsset{compat=1.18}

% ------------------ FONT & TEXT IMPROVEMENTS ------------------
\usepackage{lmodern} % Improved font rendering
\usepackage{microtype} % Better spacing and justification

% ------------------ HYPERLINKS ------------------
% \usepackage[colorlinks=true, linkcolor=blue, citecolor=red, urlcolor=blue]{hyperref} 
% Internal links in blue, citations in red, URLs in blue

% ------------------ HEADINGS & CUSTOM STYLES ------------------
\usepackage{titlesec} % Custom section formatting
\titleformat{\section}{\Large\bfseries}{\thesection}{1em}{}
\titleformat{\subsection}{\large\bfseries}{\thesubsection}{1em}{}
\titleformat{\subsubsection}{\normalsize\bfseries}{\thesubsubsection}{1em}{}

% ------------------ TITLE ------------------
\title{Documentação Farmers Market}
\author{}
\date{}

\begin{document}

\maketitle
\vspace{-67pt}

\section{User stories}
\subsection{User Story 1}
Como cliente quero criar uma conta de modo a aceder à plataforma.\\\\
\textbf{Critérios de Aceitação}:
\begin{itemize}
  \item O cliente deve selecionar o tipo de conta que está a criar.
  \item O cliente tem de introduzir o seu nome, e-mail, palavra-passe, data de nascimento e pergunta secreta.
  \item A palavra-passe deverá ter entre 8 e 16 caracteres e conter, pelo menos, um caracter especial ou um número.
  \item O sistema deve validar se o e-mail já se encontra registado.
  \item O cliente deve receber um e-mail para confirmar o registo.
\end{itemize}
\subsection{User Story 2}
Como agricultor quero criar uma conta de modo a gerir os meus produtos.\\\\
\textbf{Critérios de Aceitação}:
\begin{itemize}
  \item O agricultor deve selecionar o tipo de conta que está a criar.
  \item O agricultor tem de introduzir o seu nome, e-mail, palavra-passe, data de nascimento e pergunta secreta.
  \item A palavra-passe deverá ter entre 8 e 16 caracteres e conter, pelo menos, um caracter especial ou um número.
  \item O sistema deve validar se o e-mail já se encontra registado.
  \item O agricultor deve receber um e-mail para confirmar o registo.
\end{itemize}
\subsection{User Story 3}
Como administrador quero criar uma conta de modo a gerir o sistema.\\\\
\textbf{Critérios de Aceitação}:
\begin{itemize}
  \item O administrador deve selecionar o tipo de conta que está a criar.
  \item O administrador tem de introduzir o seu nome, e-mail, palavra-passe, data de nascimento, pergunta secreta e um código especial para validar conta de administrador.
  \item A palavra-passe deverá ter entre 8 e 16 caracteres e conter, pelo menos, um caracter especial ou um número.
  \item O sistema deve validar se o e-mail já se encontra registado.
  \item O administrador deve receber um e-mail para confirmar o registo.
\end{itemize}
\subsection{User Story 4}
Como cliente quero recuperar a palavra pass, de modo a manter o acesso à plataforma.\\\\
\textbf{Critérios de Aceitação}:
\begin{itemize}
  \item Existe um botão no menu de login para recuperar a palavra pass.
  \item O cliente deve introduzir o seu e-mail e a resposta da pergunta secreta.
  \item O cliente recebe um e-mail com uma hiperligação para alterar a palavra pass.
  \item A nova palavra-passe deverá ter entre 8 e 16 caracteres e conter, pelo menos, um caracter especial ou um número.
\end{itemize}
\subsection{User Story 5}
Como agricultor quero recuperar a palavra pass, de modo a manter o acesso à plataforma.\\\\
\textbf{Critérios de Aceitação}:
\begin{itemize}
  \item Existe um botão no menu de login para recuperar a palavra pass.
  \item O agricultor deve introduzir o seu e-mail e a resposta da pergunta secreta.
  \item O agricultor recebe um e-mail com uma hiperligação para alterar a palavra pass.
  \item A nova palavra-passe deverá ter entre 8 e 16 caracteres e conter, pelo menos, um caracter especial ou um número.
\end{itemize}
\subsection{User Story 6}
Como cliente quero fazer login na plataforma, de modo a ter acesso às minhas funcionalidades.\\\\
\textbf{Critérios de Aceitação}:
\begin{itemize}
  \item Existe um botão no canto direito do cabeçalho para fazer login.
  \item O cliente deve inserir o seu e-mail e palavra pass.
  \item O sistema deve mostrar uma mensagem de erro caso a palavra pass esteja errada.
  \item O sistema deve mostrar uma mensagem de erro caso o e-mail não esteja associado a nenhuma conta.
\end{itemize}
\subsection{User Story 7}
Como agricultor quero fazer login na plataforma, de modo a ter acesso às minhas funcionalidades.\\\\
\textbf{Critérios de Aceitação}:
\begin{itemize}
  \item Existe um botão no canto direito do cabeçalho para fazer login.
  \item O agricultor deve inserir o seu e-mail e palavra pass.
  \item O sistema deve mostrar uma mensagem de erro caso a palavra pass esteja errada.
  \item O sistema deve mostrar uma mensagem de erro caso o e-mail não esteja associado a nenhuma conta.
\end{itemize}
\subsection{User Story 8}
Como administrador quero fazer login na plataforma, de modo a ter acesso às minhas funcionalidades.\\\\
\textbf{Critérios de Aceitação}:
\begin{itemize}
  \item Existe um botão no canto direito do cabeçalho para fazer login.
  \item O administrador deve inserir o seu e-mail e palavra pass.
  \item O sistema deve mostrar uma mensagem de erro caso a palavra pass esteja errada.
  \item O sistema deve mostrar uma mensagem de erro caso o e-mail não esteja associado a nenhuma conta.
\end{itemize}
\subsection{User Story 9}
Como cliente quero comunicar com o administrador de modo a dar o conhecimento dos meus problemas.\\\\
\textbf{Critérios de Aceitação}:
\begin{itemize}
  \item No formulário o cliente deve escolher qual o tipo do problema (pedido de assistência, queixa, ou outro assunto).
  \item No formulário o cliente deve obrigatoriamente preencher o seu nome, email e uma descrição do sucedido.
  \item No formulário o cliente tem a opção de anexar ficheiros.
  \item Após submeter o formulário, o utilizador recebe uma confirmação visual com uma mensagem a dizer “Formulário submetido com sucesso.”.
\end{itemize}
\subsection{User Story 10}
Como administrador quero responder a formulários de modo a resolver problemas dos clientes.\\\\
\textbf{Critérios de Aceitação}:
\begin{itemize}
  \item A resposta será enviada para o e-mail do cliente.
  \item O administrador pode alterar o estado do problema (em curso ou resolvido).
\end{itemize}
\subsection{User Story 11}
Como cliente quero visualizar os meus problemas, de modo a verificar o seu estado.\\\\
\textbf{Critérios de Aceitação}:
\begin{itemize}
  \item Existe um botão na área de cliente para ver o histórico de problemas.
  \item O histórico de problemas deve estar organizado de forma cronológica.
  \item Os problemas têm uma descrição que diz se o mesmo está resolvido ou em curso.
\end{itemize}
\subsection{User Story 12}
Como cliente quero pesquisar por agricultor de modo a comprar diretamente aos meus agricultores favoritos.\\\\
\textbf{Critérios de Aceitação}:
\begin{itemize}
  \item Os resultados devem ser apresentados com nome, imagem e localização.
  \item Deve de existir a opção de filtrar os agricultores por distrito.
  \item Deve de existir a opção de ordenar os agricultores por ordem alfabética ou de rating.
  \item A página de cada agricultor deverá mostrar todos os produtos desse agricultor que se encontrem disponíveis no momento.
  \item Cada produto deve exibir: nome, imagem, preço por unidade, quantidade disponível e descrição.
\end{itemize}
\subsection{User Story 13}
Como cliente quero pesquisar por categoria de produto de modo a comprar os produtos que necessito.\\\\
\textbf{Critérios de Aceitação}:
\begin{itemize}
  \item Devem aparecer as opções com as várias categorias de produtos disponíveis.
  \item Ao selecionar a categoria pretendida, os resultados devem ser apresentados com nome, imagem, preço por unidade e quantidade disponível.
  \item Deve existir a opção de ordenar os produtos por ordem alfabética ou de preço, ascendente ou descendente.
  \item As categorias existentes são legumes, fruta, mel, cereais e doces artesanais.
\end{itemize}
\subsection{User Story 14}
Como agricultor quero registar produtos de modo a poder vendê-los.\\\\
\textbf{Critérios de Aceitação}:
\begin{itemize}
  \item O sistema deve apresentar uma lista de categorias.
  \item O agricultor escolhe a categoria à qual quer adicionar 
  \item Ao adicionar um produto, o agricultor insere o nome, a quantidade, preço por unidade uma imagem e uma descrição facultativa.
\end{itemize}
\subsection{User Story 15}
Como agricultor quero alterar os meus produtos, de modo a mantê-los disponíveis.\\\\
\textbf{Critérios de Aceitação}:
\begin{itemize}
  \item A quantidade disponível, preço, imagem e descrição de cada produto deve ser visível para o agricultor. sistema deve apresentar uma lista de categorias.
  \item Ao selecionar um dos seus produtos, o agricultor pode alterar a sua quantidade disponível.
  \item Ao selecionar um dos seus produtos, o agricultor pode alterar o seu preço.
  \item Ao selecionar um dos seus produtos, o agricultor pode alterar a sua imagem.
  \item Ao selecionar um dos seus produtos, o agricultor pode alterar a sua descrição.
\end{itemize}
\subsection{User Story 16}
Como cliente quero ver dicas de produção de modo a aprender a cultivar produtos biológicos.\\\\
\textbf{Critérios de Aceitação}:
\begin{itemize}
  \item Cada dica é acompanhada de uma descrição explicativa e imagem facultativa.
  \item Existe um botão chamados “Dicas de cultivo” no cabeçalho.
\end{itemize}
\subsection{User Story 17}
Como cliente quero avaliar um agricultor, de modo a ajudar outros utilizadores na tomada de decisão.\\\\
\textbf{Critérios de Aceitação}:
\begin{itemize}
  \item A avaliação só pode ser feita por clientes registados e autenticados.
  \item A avaliação deve ser feita de 0 a 5 estrelas.
  \item Após submeter a avaliação, o utilizador recebe uma confirmação visual (ex: mensagem de sucesso).
  \item O cliente só pode avaliar o agricultor após uma compra.
\end{itemize}
\subsection{User Story 18}
Como cliente quero pagar para poder comprar produtos online.\\\\
\textbf{Critérios de Aceitação}:
\begin{itemize}
  \item O cliente final tem disponível três métodos de pagamento distintos que pode selecionar: MBWay, Visa/Cartão Crédito e transferência bancária.
  \item O total a ser pago deve ser exibido na página de pagamento.
  \item O sistema deve dar uma indicação clara quanto à validade dos dados inseridos no método de pagamento.
  \item A transação deve ser processada com confirmação de sucesso ou erro.
  \item Os dados bancários devem ser encriptados usando protocolo seguro durante o processo de pagamento.
  \item Após a submissão dos dados de pagamento o sistema deve aguardar a validação dos mesmos pela entidade responsável antes de permitir finalizar o pedido.
  \item No caso de falha no pagamento, o sistema deve apresentar uma mensagem clara informando o motivo do erro.
  \item Durante o processo de pagamento, o cliente deve visualizar uma indicação de que o pagamento está a ser processado (uma roda).
  \item Apenas clientes registados podem fazer compras.
\end{itemize}
\subsection{User Story 19}
Como agricultor quero visualizar encomendas a satisfazer, de modo a entregar as mesmas.\\\\
\textbf{Critérios de Aceitação}:
\begin{itemize}
  \item Todos os agricultores deverão ser notificados via e-mail, de todas as encomendas que foram processadas nesse dia, incluindo o comprador, morada do comprador, produto(s) e quantidade(s).
  \item As encomendas estão disponíveis na área do agricultor.
\end{itemize}
\subsection{User Story 20}
Como cliente quero ver o perfil de um agricultor, de modo a saber as suas informações.\\\\
\textbf{Critérios de Aceitação}:
\begin{itemize}
  \item No perfil do agricultor devem aparecer as suas técnicas de produção sustentável.
  \item No perfil do agricultor devem aparecer os produtos que o mesmo está a vender.
  \item No perfil do agricultor deve aparecer a data até à qual aceita encomendas e o dia em que faz as entregas.
  \item No perfil do agricultor deve aparecer o nome, localização e a média de classificação dada pelos clientes.
\end{itemize}
\subsection{User Story 21}
Como cliente quero adicionar produtos a um carrinho de compras, de modo a saber quanto vou pagar antes de concluir a compra.\\\\
\textbf{Critérios de Aceitação}:
\begin{itemize}
  \item O carrinho deve permitir alterar a quantidade de um produto já adicionado.
  \item O sistema deve calcular automaticamente o total com base na quantidade e preço de cada produto.
  \item O carrinho deve ser acessível a partir de qualquer página do website.
  \item O carrinho deve guardar os produtos mesmo que o utilizador mude de página.
  \item O botão “Finalizar Compra” só aparece se houver produtos no carrinho.
  \item O carrinho deve estar acessível através de um ícone fixo (no canto superior direito).
\end{itemize}
\subsection{User Story 22}
Como cliente quero remover artigos do meu carrinho, de modo a não comprar o que não quero.\\\\
\textbf{Critérios de Aceitação}:
\begin{itemize}
  \item O cliente pode esvaziar todo o carrinho com um só clique.
  \item Cada artigo deve ter um botão para o remover.
\end{itemize}
\subsection{User Story 23}
Como cliente quero escolher o idioma do website, de modo a conseguir compreender facilmente todas as informações.\\\\
\textbf{Critérios de Aceitação}:
\begin{itemize}
  \item Qualquer utilizador do site pode escolher o idioma entre português e inglês.
  \item Deve haver um botão com as bandeiras de Portugal e Reino Unido para selecionar o idioma.
  \item A seleção de idioma deverá ser acessível a partir de qualquer página do website.
  \item Ao alterar o idioma, o website deve manter o utilizador na mesma página, apenas atualizando o idioma.
  \item O idioma escolhido deve persistir durante a navegação.
\end{itemize}
\subsection{User Story 24}
Como cliente quero apenas visualizar produtos que se encontrem disponíveis, de modo a evitar comprar produtos sem stock.\\\\
\textbf{Critérios de Aceitação}:
\begin{itemize}
  \item O sistema deve atualizar manter o stock de todos os produtos atualizado.
  \item Quando não existir stock de um produto o mesmo deve ser eliminado do catálogo.
  \item Quando for adicionado stock o produto torna a estar novamente visível.
\end{itemize}
\subsection{User Story 25}
Como cliente quero comunicar com os agricultores, de modo a obter informações sobre os seus produtos.\\\\
\textbf{Critérios de Aceitação}:
\begin{itemize}
  \item No perfil dos agricultores deve existir um botão para comunicar com o mesmo.
  \item O cliente deve introduzir o seu nome, assunto e e-mail.
  \item A mensagem deve ter, no máximo, 500 caracteres.
\end{itemize}
\subsection{User Story 26}
Como agricultor quero receber as mensagens dos clientes, de modo a comunicar com os mesmos.\\\\
\textbf{Critérios de Aceitação}:
\begin{itemize}
  \item Os agricultores recebem por e-mail as mensagens dos clientes.
\end{itemize}

\section{Definition of Done}
\begin{itemize}
  \item Todas as user stories devem ser realizadas, sendo que alguns critérios de aceitação, previamente acordados em grupo, não serão realizados.
  \item Os diferentes membros da equipa devem concordaar com as user stories a serem desenvolvidas, quer no sprint planning, quer as user stories que são adicionadas durante o sprint.
  \item Todas as funcionalidades implementadas foram testadas pelos diferentes membros da equipa.
  \item Todas as funcionalidades implementadas foram apresentadas aos membros da equipa que as aprovaram.
  \item O Sprint Review foi preparado.
\end{itemize}
\end{document}


