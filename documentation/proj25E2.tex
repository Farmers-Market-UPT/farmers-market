\documentclass[a4paper,11pt]{article}  

% ------------------ ENCODING & LANGUAGE ------------------
\usepackage[utf8]{inputenc} % UTF-8 encoding
\usepackage[T1]{fontenc}    % Proper font encoding for special characters
\usepackage[portuguese]{babel} % Portuguese hyphenation & translations

% ------------------ PAGE LAYOUT ------------------
\usepackage[left=2cm, right=2cm, top=2cm, bottom=2cm]{geometry} % Custom margins
\usepackage{setspace}   % Control line spacing
\onehalfspacing         % 1.5 line spacing for better readability
\usepackage{parskip}    % Adds space between paragraphs instead of indentation

% ------------------ MATH & SYMBOLS ------------------
\usepackage{amsmath, amssymb, amsthm} % Core math packages
\usepackage{mathtools} % Extends amsmath (e.g., better equation numbering)
\usepackage{bm} % Bold symbols in math mode
\usepackage{siunitx} % SI units, e.g., \SI{9.8}{m/s^2}

% ------------------ GRAPHICS & FIGURES ------------------
\usepackage{graphicx} % Include images
\usepackage{float} % Allow pictures to stay where I put them
\usepackage{xcolor} % Colors for text and diagrams
\usepackage{tikz} % Drawing graphs and diagrams
\usepackage{tikz}
\usetikzlibrary{decorations.pathreplacing}
\usepackage{pgfplots} % Plot functions and data
\pgfplotsset{compat=1.18}

% ------------------ FONT & TEXT IMPROVEMENTS ------------------
\usepackage{lmodern} % Improved font rendering
\usepackage{microtype} % Better spacing and justification

% ------------------ HYPERLINKS ------------------
% \usepackage[colorlinks=true, linkcolor=blue, citecolor=red, urlcolor=blue]{hyperref} 
% Internal links in blue, citations in red, URLs in blue

% ------------------ HEADINGS & CUSTOM STYLES ------------------
\usepackage{titlesec} % Custom section formatting
\titleformat{\section}{\Large\bfseries}{\thesection}{1em}{}
\titleformat{\subsection}{\large\bfseries}{\thesubsection}{1em}{}
\titleformat{\subsubsection}{\normalsize\bfseries}{\thesubsubsection}{1em}{}

% ------------------ TITLE ------------------
\title{Documentação Farmers Market}
\author{}
\date{}

\begin{document}

\maketitle
\vspace{-67pt}

\section{Product Backlog}
\subsection{User Story 1}
Como cliente quero criar uma conta de modo a aceder à plataforma.\\\\
\textbf{Critérios de Aceitação}:
\begin{itemize}
  \item O cliente deve selecionar o tipo de conta que está a criar.
  \item O cliente tem de introduzir o seu nome, e-mail, palavra-passe, data de nascimento, localização e pergunta secreta.
  \item A palavra-passe deverá ter entre 8 e 16 caracteres e conter, pelo menos, um caracter especial ou um número.
  \item Caso a palavra-passe não esteja de acordo com os critérios estabelecidos, aparece uma mensagem de erro.
  \item Caso a data de nascimento seja no futuro, aparece uma mensagem de erro.
  \item Caso o e-mail já se encontre registado, aparece uma mensagem de erro.
  \item O cliente deve receber um e-mail para confirmar o registo.
\end{itemize}
\subsection{User Story 2}
Como agricultor quero criar uma conta de modo a gerir os meus produtos.\\\\
\textbf{Critérios de Aceitação}:
\begin{itemize}
  \item O agricultor deve selecionar o tipo de conta que está a criar.
  \item O agricultor tem de introduzir o seu nome, e-mail, palavra-passe, data de nascimento, localização e pergunta secreta.
  \item A palavra-passe deverá ter entre 8 e 16 caracteres e conter, pelo menos, um caracter especial ou um número.
  \item Caso a palavra-passe não esteja de acordo com os critérios estabelecidos, aparece uma mensagem de erro.
  \item Caso a data de nascimento seja no futuro, aparece uma mensagem de erro.
  \item Caso o e-mail já se encontre registado, aparece uma mensagem de erro.
  \item O agricultor deve receber um e-mail para confirmar o registo.
\end{itemize}
\subsection{User Story 3}
Como administrador quero criar uma conta de modo a gerir o sistema.\\\\
\textbf{Critérios de Aceitação}:
\begin{itemize}
  \item O administrador deve selecionar o tipo de conta que está a criar.
  \item O administrador tem de introduzir o seu nome, e-mail, palavra-passe, data de nascimento, pergunta secreta e um código especial para administradores.
  \item A palavra-passe deverá ter entre 8 e 16 caracteres e conter, pelo menos, um caracter especial ou um número.
  \item Caso a palavra-passe não esteja de acordo com os critérios estabelecidos, aparece uma mensagem de erro.
  \item Caso a data de nascimento seja no futuro, aparece uma mensagem de erro.
  \item Caso o e-mail já se encontre registado, aparece uma mensagem de erro.
  \item Caso o código especial esteja errado, aparece uma mensagem de erro.
  \item O administrador deve receber um e-mail para confirmar o registo.
\end{itemize}
\subsection{User Story 4}
Como agricultor quero adicionar as minhas técnicas de cultivo biológicas ao meu perfil de modo a mostrar como são cultivados os produtos que vendo.\\\\
\textbf{Critérios de Aceitação}:
\begin{itemize}
  \item Deve de existir um botão que permita ao agricultor adicionar técnicas de cultivo biológico.
  \item Cada técnica deve ter um nome e ser acompanha de uma descrição.
  \item A descrição deverá ter um máximo de 1000 caracteres.
  \item Caso a descrição exceda o limite de caracteres aparece uma mensagem de erro.
  \item Os campos nome e descrição têm de ser preenchidos obrigatoriamente.
  \item Caso um dos campos obrigatórios não esteja preenchido aparece uma mensagem de erro.
\end{itemize}
\subsection{User Story 5}
Como cliente quero recuperar a palavra-passepass, de modo a manter o acesso à plataforma.\\\\
\textbf{Critérios de Aceitação}:
\begin{itemize}
  \item Existe um botão no menu de login para recuperar a palavra-passe.
  \item O cliente deve introduzir o seu e-mail e a resposta da pergunta secreta.
  \item O cliente recebe um e-mail com uma hiperligação para alterar a palavra-passe.
  \item A nova palavra-passe deverá ter entre 8 e 16 caracteres e conter, pelo menos, um caracter especial ou um número.
  \item Caso o e-mail não se encontre registado aparece uma mensagem de erro.
  \item Caso a resposta da pergunta secreta esteja errada aparece uma mensagem de erro.
  \item Caso a palavra-passe não esteja de acordo com os critérios estabelecidos aparece uma mensagem de erro.
  \item Após trocar a palavra-passe aparece uma mensagem de sucesso.
\end{itemize}
\subsection{User Story 6}
Como agricultor quero recuperar a palavra-passe, de modo a manter o acesso à plataforma.\\\\
\textbf{Critérios de Aceitação}:
\begin{itemize}
  \item Existe um botão no menu de login para recuperar a palavra-passe.
  \item O agricultor deve introduzir o seu e-mail e a resposta da pergunta secreta.
  \item O agricultor recebe um e-mail com uma hiperligação para alterar a palavra-passe.
  \item A nova palavra-passe deverá ter entre 8 e 16 caracteres e conter, pelo menos, um caracter especial ou um número.
  \item Caso o e-mail não se encontre registado aparece uma mensagem de erro.
  \item Caso a resposta da pergunta secreta esteja errada aparece uma mensagem de erro.
  \item Caso a palavra-passe não esteja de acordo com os critérios estabelecidos, aparece uma mensagem de erro.
  \item Após trocar a palavra-passe aparece uma mensagem de sucesso.
\end{itemize}
\subsection{User Story 7}
Como cliente quero fazer login na plataforma, de modo a ter acesso às minhas funcionalidades.\\\\
\textbf{Critérios de Aceitação}:
\begin{itemize}
  \item Existe um botão no canto direito do cabeçalho para fazer login.
  \item O cliente deve inserir o seu e-mail e palavra-passe.
  \item O sistema deve mostrar uma mensagem de erro caso a palavra-passe esteja errada.
  \item O sistema deve mostrar uma mensagem de erro caso o e-mail não esteja associado a nenhuma conta.
\end{itemize}
\subsection{User Story 8}
Como agricultor quero fazer login na plataforma, de modo a ter acesso às minhas funcionalidades.\\\\
\textbf{Critérios de Aceitação}:
\begin{itemize}
  \item Existe um botão no canto direito do cabeçalho para fazer login.
  \item O agricultor deve inserir o seu e-mail e palavra-passe.
  \item O sistema deve mostrar uma mensagem de erro caso a palavra-passe esteja errada.
  \item O sistema deve mostrar uma mensagem de erro caso o e-mail não esteja associado a nenhuma conta.
\end{itemize}
\subsection{User Story 9}
Como administrador quero fazer login na plataforma, de modo a ter acesso às minhas funcionalidades.\\\\
\textbf{Critérios de Aceitação}:
\begin{itemize}
  \item Existe um botão no canto direito do cabeçalho para fazer login.
  \item O administrador deve inserir o seu e-mail e palavra-passe.
  \item O sistema deve mostrar uma mensagem de erro caso a palavra-passe esteja errada.
  \item O sistema deve mostrar uma mensagem de erro caso o e-mail não esteja associado a nenhuma conta.
\end{itemize}
\subsection{User Story 10}
Como agricultor quero mudar a minha questão secreta a qualquer momento de modo a garantir que sei a resposta à pergunta.\\\\
\textbf{Critérios de Aceitação}:
\begin{itemize}
  \item Esta opção deverá aparecer no menu do agricultor após o login.
  \item De modo a validar esta alteração, o agricultor deve introduzir novamente a sua palavra-passe.
  \item Caso a palavra-passe esteja errada aparece uma mensagem de erro.
  \item Após alterar a questão secreta aparece uma mensagem de sucesso.
\end{itemize}
\subsection{User Story 11}
Como cliente quero mudar a minha questão secreta a qualquer momento de modo a garantir que sei a resposta à pergunta.\\\\
\textbf{Critérios de Aceitação}:
\begin{itemize}
  \item Esta opção deverá aparecer no menu do cliente após o login.
  \item De modo a validar esta alteração, o cliente deve introduzir novamente a sua palavra-pass.
  \item Caso a palavra-passe esteja errada aparece uma mensagem de erro.
  \item Após alterar a questão secreta aparece uma mensagem de sucesso.
\end{itemize}
\subsection{User Story 12}
Como administrador quero mudar a minha questão secreta a qualquer momento de modo a garantir que sei a resposta à pergunta.\\\\
\textbf{Critérios de Aceitação}:
\begin{itemize}
  \item Esta opção deverá aparecer no menu do administrador após o login.
  \item De modo a validar esta alteração, o administrador deve introduzir novamente a sua palavra-passe.
  \item Caso a palavra-passe esteja errada aparece uma mensagem de erro.
  \item Após alterar a questão secreta aparece uma mensagem de sucesso.
\end{itemize}
\subsection{User Story 13}
Como cliente quero comunicar com o administrador de modo a dar o conhecimento dos meus problemas.\\\\
\textbf{Critérios de Aceitação}:
\begin{itemize}
  \item No formulário o cliente deve escolher qual o tipo do problema (pedido de assistência, queixa, ou outro assunto).
  \item No formulário o cliente deve obrigatoriamente preencher o seu nome, e-mail e uma descrição do sucedido.
  \item Caso algum dos campos obrigatórios não se encontre preenchido aparece uma mensagem de erro.
  \item No formulário o cliente tem a opção de anexar ficheiros.
  \item Após submeter o formulário, o utilizador recebe uma confirmação visual com uma mensagem a dizer “Formulário submetido com sucesso.”.
\end{itemize}
\subsection{User Story 14}
Como administrador quero responder a formulários de modo a resolver problemas dos clientes.\\\\
\textbf{Critérios de Aceitação}:
\begin{itemize}
  \item No menu do administrador existe um botão que mostra os problemas dos clientes.
  \item O administrador pode filtrar os problemas pelo seu estado.
  \item A resposta será enviada para o e-mail do cliente.
  \item O administrador pode alterar o estado do problema (em curso ou resolvido).
\end{itemize}
\subsection{User Story 15}
Como cliente quero visualizar os meus problemas, de modo a verificar o seu estado.\\\\
\textbf{Critérios de Aceitação}:
\begin{itemize}
  \item Existe um botão na área de cliente para ver o histórico de problemas.
  \item O histórico de problemas deve estar organizado de forma cronológica.
  \item Os problemas têm uma descrição que diz se o mesmo está resolvido ou em curso.
\end{itemize}
\subsection{User Story 16}
Como cliente quero pesquisar por agricultor de modo a comprar diretamente aos meus agricultores favoritos.\\\\
\textbf{Critérios de Aceitação}:
\begin{itemize}
  \item Os resultados devem ser apresentados com nome e imagem.
  \item Os resultados devem ser apresentados em ordem alfabética.
\end{itemize}
\subsection{User Story 17}
Como cliente quero pesquisar por categoria de produto de modo a comprar os produtos que necessito.\\\\
\textbf{Critérios de Aceitação}:
\begin{itemize}
  \item Devem aparecer as opções com as várias categorias de produtos disponíveis.
  \item Ao selecionar a categoria pretendida, os resultados devem ser apresentados com nome, imagem, preço por unidade e quantidade disponível.
  \item Deve existir a opção de ordenar os produtos por ordem alfabética ou de preço, ascendente ou descendente.
  \item As categorias existentes são legumes, fruta e cereais.
\end{itemize}
\subsection{User Story 18}
Como agricultor quero registar produtos de modo a poder vendê-los.\\\\
\textbf{Critérios de Aceitação}:
\begin{itemize}
  \item O sistema deve apresentar uma lista de categorias.
  \item O agricultor escolhe a categoria à qual quer adicionar 
  \item Ao adicionar um produto, o agricultor insere obrigatoriamente o nome, a quantidade, preço por unidade e uma imagem.
  \item Caso algum dos campos obrigatórios esteja vazio aparece uma mensagem de erro.
\end{itemize}
\subsection{User Story 19}
Como agricultor quero alterar os meus produtos, de modo a mantê-los disponíveis.\\\\
\textbf{Critérios de Aceitação}:
\begin{itemize}
  \item A quantidade disponível, preço, imagem e descrição de cada produto deve ser visível para o agricultor.
  \item Ao selecionar um dos seus produtos, o agricultor pode alterar a sua quantidade disponível.
  \item Ao selecionar um dos seus produtos, o agricultor pode alterar o seu preço.
  \item Ao selecionar um dos seus produtos, o agricultor pode alterar a sua imagem.
  \item Caso haja alguma alteração aparece uma mensagem de sucesso.
\end{itemize}
\subsection{User Story 20}
Como cliente quero ver dicas de produção de modo a aprender a cultivar produtos biológicos.\\\\
\textbf{Critérios de Aceitação}:
\begin{itemize}
  \item Cada dica é acompanhada de um nome, descrição explicativa e uma imagem.
  \item Caso algum dos campos obrigatórios não esteja preenchido aparece uma mensagem de erro.
  \item Existe um botão chamado “Dicas de cultivo” no cabeçalho.
\end{itemize}
\subsection{User Story 21}
Como cliente quero avaliar um agricultor, de modo a ajudar outros utilizadores na tomada de decisão.\\\\
\textbf{Critérios de Aceitação}:
\begin{itemize}
  \item A avaliação deve ser feita de 0 a 5 estrelas.
  \item Após submeter a avaliação, o utilizador recebe uma mensagem de sucesso.
  \item O cliente só pode avaliar o agricultor após uma compra.
\end{itemize}
\subsection{User Story 22}
Como cliente quero pagar para poder comprar produtos online.\\\\
\textbf{Critérios de Aceitação}:
\begin{itemize}
  \item O cliente tem disponível três métodos de pagamento distintos que pode selecionar: Paypal, Visa e transferência bancária.
  \item O total a ser pago deve ser exibido na página de pagamento.
  \item O sistema deve dar uma indicação clara quanto à validade dos dados inseridos no método de pagamento.
  \item A transação deve ser processada com confirmação de sucesso ou erro.
  \item Os dados bancários devem ser encriptados usando protocolo seguro durante o processo de pagamento.
  \item Após a submissão dos dados de pagamento o sistema deve aguardar a validação dos mesmos pela entidade responsável antes de permitir finalizar o pedido.
  \item No caso de falha no pagamento, o sistema deve apresentar uma mensagem clara informando o motivo do erro.
  \item Durante o processo de pagamento, o cliente deve visualizar uma indicação de que o pagamento está a ser processado (uma roda).
\end{itemize}
\subsection{User Story 23}
Como agricultor quero visualizar encomendas a satisfazer, de modo a entregar as mesmas.\\\\
\textbf{Critérios de Aceitação}:
\begin{itemize}
  \item Os agricultores devem ser diariamente notificados via e-mail de todas as encomendas processadas para si.
  \item As encomendas estão disponíveis no menu do agricultor.
  \item Ao consultar o seu histórico de vendas, nos detalhes de cada encomenda deve aparecer o comprador, morada do comprador, produto(s), quantidade(s) e preço total.
  \item No histórico de encomendas do agricultor deverá aparecer somente os produtos que ele vendeu em cada compra.
\end{itemize}
\subsection{User Story 24}
Como cliente quero ver o perfil de um agricultor, de modo a saber as suas informações.\\\\
\textbf{Critérios de Aceitação}:
\begin{itemize}
  \item No perfil do agricultor devem aparecer as suas técnicas de produção sustentável.
  \item No perfil do agricultor devem aparecer os produtos que o mesmo está a vender.
  \item Cada produto deve exibir: nome, imagem, preço por unidade e quantidade disponível.
  \item No perfil do agricultor deve aparecer o nome, localização e a média de classificação dada pelos clientes.
  \item Existe um botão que permite adicionar produtos selecionados ao carrinho.
\end{itemize}
\subsection{User Story 25}
Como cliente quero adicionar produtos a um carrinho de compras, de modo a saber quanto vou pagar antes de concluir a compra.\\\\
\textbf{Critérios de Aceitação}:
\begin{itemize}
  \item O carrinho deve permitir alterar a quantidade de um produto já adicionado.
  \item O sistema deve calcular automaticamente o total com base na quantidade e preço de cada produto.
  \item O carrinho deve ser acessível a partir de qualquer página do website.
  \item O carrinho deve guardar os produtos mesmo que o cliente faça logout.
  \item Existe um botão para finalizar compra.
  \item Ao finalizar a compra, aparece uma mensagem de erro caso o carrinho esteja vazio.
\end{itemize}
\subsection{User Story 26}
Como cliente quero remover artigos do meu carrinho, de modo a não comprar o que não quero.\\\\
\textbf{Critérios de Aceitação}:
\begin{itemize}
  \item Existe um botão para remover todos os artigos do carrinho.
  \item Existe um botão para remover artigos selecionados.
  \item Caso nenhum artigo esteja selecionado ao carregar no botão aparece uma mensagem de erro.
\end{itemize}
\subsection{User Story 27}
Como cliente quero escolher o idioma do website, de modo a conseguir compreender facilmente todas as informações.\\\\
\textbf{Critérios de Aceitação}:
\begin{itemize}
  \item Os idiomas disponíveis são português e inglês.
  \item Deve haver um botão com as bandeiras de Portugal e Reino Unido para selecionar o idioma.
  \item A seleção de idioma deverá ser acessível a partir de qualquer página do website.
  \item Ao alterar o idioma, o website deve manter o cliente na mesma página, apenas atualizando o idioma.
  \item O idioma escolhido deve persistir durante a navegação.
\end{itemize}
\subsection{User Story 28}
Como agricultor quero escolher o idioma do website, de modo a conseguir compreender facilmente todas as informações.\\\\
\textbf{Critérios de Aceitação}:
\begin{itemize}
  \item Os idiomas disponíveis são português e inglês.
  \item Deve haver um botão com as bandeiras de Portugal e Reino Unido para selecionar o idioma.
  \item A seleção de idioma deverá ser acessível a partir de qualquer página do website.
  \item Ao alterar o idioma, o website deve manter o agricultor na mesma página, apenas atualizando o idioma.
  \item O idioma escolhido deve persistir durante a navegação.
\end{itemize}
\subsection{User Story 29}
Como administrador quero escolher o idioma do website, de modo a conseguir compreender facilmente todas as informações.\\\\
\textbf{Critérios de Aceitação}:
\begin{itemize}
  \item Os idiomas disponíveis são português e inglês.
  \item Deve haver um botão com as bandeiras de Portugal e Reino Unido para selecionar o idioma.
  \item A seleção de idioma deverá ser acessível a partir de qualquer página do website.
  \item Ao alterar o idioma, o website deve manter o administrador na mesma página, apenas atualizando o idioma.
  \item O idioma escolhido deve persistir durante a navegação.
\end{itemize}
\subsection{User Story 30}
Como cliente quero apenas visualizar produtos que se encontrem disponíveis, de modo a evitar comprar produtos sem stock.\\\\
\textbf{Critérios de Aceitação}:
\begin{itemize}
  \item O sistema deve atualizar manter o stock de todos os produtos atualizado.
  \item Quando não existir stock de um produto o mesmo deve ser eliminado do catálogo.
  \item Quando for adicionado stock o produto torna a estar novamente visível.
\end{itemize}
\subsection{User Story 31}
Como cliente quero comunicar com os agricultores, de modo a obter informações sobre os seus produtos.\\\\
\textbf{Critérios de Aceitação}:
\begin{itemize}
  \item No perfil dos agricultores deve existir um botão para comunicar com o mesmo.
  \item O cliente deve introduzir obrigatoriamente o seu nome, assunto e e-mail.
  \item Caso algum dos campos obrigatórios não seja preenchido aparece uma mensagem de erro.
  \item A mensagem deve ter, no máximo, 500 caracteres.
  \item Caso a mensagem exceda o limite de caracteres aparece uma mensagem de erro.
\end{itemize}
\subsection{User Story 32}
Como agricultor quero receber as mensagens dos clientes, de modo a comunicar com os mesmos.\\\\
\textbf{Critérios de Aceitação}:
\begin{itemize}
  \item Os agricultores recebem por e-mail as mensagens dos clientes.
  \item Dado que o e-mail do cliente é visível, o agricultor pode responder através de e-mail.
\end{itemize}
\subsection{User Story 33}
Como cliente quero aceder ao Farmers Market através de uma aplicação móvel para poder comprar produtos de forma prática no telemóvel.\\\\
\textbf{Critérios de Aceitação}:
\begin{itemize}
  \item A aplicação deve estar disponível para Android e iOS.
  \item A interface deve ser adaptada a ecrãs de dispositivos móveis.
  \item Todas as funcionalidades do site devem estar disponíveis na app.
\end{itemize}
\subsection{User Story 34}
Como agricultor quero aceder ao Farmers Market através de uma aplicação móvel para poder vender produtos de forma prática no telemóvel.\\\\
\textbf{Critérios de Aceitação}:
\begin{itemize}
  \item A aplicação deve estar disponível para Android e iOS.
  \item A interface deve ser adaptada a ecrãs de dispositivos móveis.
  \item Todas as funcionalidades do site devem estar disponíveis na app.
\end{itemize}
\subsection{User Story 35}
Como administrador quero adicionar dicas de produção de modo a partilhar conhecimento sobre cultivo de produtos biológicos.\\\\
\textbf{Critérios de Aceitação}:
\begin{itemize}
  \item Existe um botão chamado "Adicionar dicas de cultivo" no menu do administrador.
  \item Cada dica é obrigatoriamente acompanhada de um nome, descrição explicativa e imagem.
  \item Caso algum campo obrigatório não esteja preenchido aparece uma mensagem de erro.
  \item Caso a dica seja adicionada aparece uma mensagem de sucesso.
\end{itemize}
\subsection{User Story 36}
Como administrador quero remover utilizadores de modo a garantir a segurança da comunidade.\\\\
\textbf{Critérios de Aceitação}:
\begin{itemize}
  \item Caso algum utilizador apresente um comportamento não cívico o administrador tem o direito de lhe retirar o acesso à plataforma.
  \item Existe uma lista de utilizadores à qual o administrador tem acesso.
  \item Existe uma lista de e-mails com os utilizadores removidos, à qual só o administrador tem acesso.
  \item Na lista de utilizadores tem um botão “Remover”.
  \item O administrador deve justificar a causa da remoção do utilizador.
  \item O utilizador cujo acesso foi removido deve receber um e-mail a informá-lo assim como a respetiva justificação.
  \item Os registos de encomendas deste utilizadores não devem ser removidos da base de dados.
  \item Um utilizador que cujo acesso seja removido deve passar a ser invisível para os restantes utilizadores.
\end{itemize}
\subsection{User Story 37}
Como administrador quero distribuir os lucros das vendas pelos agricultores de modo  a que cada um receba a sua parte correspondente.\\\\
\textbf{Critérios de Aceitação}:
\begin{itemize}
  \item No menu do administrador aparece uma opção para consultar as diferentes encomendas realizadas.
  \item O administrador deve selecionar o ano e mês em que quer ver as encomendas.
  \item Ao selecionar o espaço de tempo a consultar aparecerá a lista de encomendas realizadas.
  \item As diferentes encomendas aparecem como “validada” ou “por validar”.
  \item Ao selecionar cada encomenda por validar aparecerão os diferentes agricultores a quem foram feitas encomendas, os respetivos itens vendidos e o total a receber por essa encomenda.
  \item O administrador deve assim confirmar o valor a atribuir a cada agricultor.
  \item Ao confirmar o valor a ser atribuido, a encomenda passa ao estado “validada”.
\end{itemize}
\subsection{User Story 38}
Como cliente quero fazer logout da minha conta de modo a garantir a segurança dos meus dados.\\\\
\textbf{Critérios de Aceitação}:
\begin{itemize}
  \item No menu do cliente deve existir um botão para fazer logout.
\end{itemize}
\subsection{User Story 39}
Como agricultor quero fazer logout da minha conta de modo a garantir a segurança dos meus dados.\\\\
\textbf{Critérios de Aceitação}:
\begin{itemize}
  \item No menu do agricultor deve existir um botão para fazer logout.
\end{itemize}
\subsection{User Story 40}
Como administrador quero fazer logout da minha conta de modo a garantir a segurança dos meus dados.\\\\
\textbf{Critérios de Aceitação}:
\begin{itemize}
  \item No menu do administrador deve existir um botão para fazer logout.
\end{itemize}
\subsection{User Story 41}
Como administrador quero recuperar a palavra-passe, de modo a manter o acesso à plataforma.\\\\
\textbf{Critérios de Aceitação}:
\begin{itemize}
  \item Existe um botão no menu de login para recuperar a palavra-passe.
  \item O administrador deve introduzir o seu e-mail e a resposta da pergunta secreta.
  \item O administrador recebe um e-mail com uma hiperligação para alterar a palavra-passe.
  \item A nova palavra-passe deverá ter entre 8 e 16 caracteres e conter, pelo menos, um caractere especial ou um número.
  \item Caso o e-mail não esteja registado aparece uma mensagem de erro.
  \item Caso a resposta da pergunta secreta esteja errada aparece uma mensagem de erro.
  \item Caso a palavra-passe não esteja de acordo com os critérios estabelecidos aparece uma mensagem de erro.
  \item Após trocar a palavra-passe aparece uma mensagem de sucesso.
\end{itemize}

\vspace{30pt}
\section{Sprint Backlogs}
Em cada um dos sprints, durante os respetivos sprint plannings, comprometemo-nos a implementar as seguintes User Stories:
\begin{itemize}
  \item \textbf{1º Sprint}: US01, US02, US07, US08, US16, US17, US18 e US24.
  \item \textbf{2º Sprint}: US19, US20, US22, US23, US25, US26, US30 e US35.
\end{itemize}
  
\vspace{130pt}
\section{First Sprint Retrospective}
Durante o sprint planning comprometemo-nos a realizar as user stories nº 1, 2, 7, 8, 16, 17, 18 e 24 durante um sprint de 2 semanas.\\\\
Neste sprint, fomos não só capazes de implementar todas as user stories às quais nos comprometemos, como também conseguimos ainda implementar algumas extra, tais como as user stories nº 3, 4 e 9.\\\\
Do ponto de vista do desempenho enquanto grupo, todos os membros concordam que o sprint decorreu sem conflitos ou atritos. Relativamente à comunicação, tentou-se sempre ouvir os diferentes elementos do grupo e as suas opiniões e sugestões. Todos os elementos do grupo se mostraram recetivos a mudanças de planos e, como mencionado anteriormente, às diferentes opiniões de cada um. No entanto, foi sugerido em grupo que, de modo a melhorar a nossa comunicação, deveríamos ir atualizando os restantes membros sobre o que estamos a fazer e o que já foi feito.\\\\
Existiram alguns critérios de aceitação de determinadas user stories que não puderam ser cumpridos, devido a competências ainda não adiquiridas, assim como em consequência de uma gestão de tempo a ser melhorada. Porém, um dos nossos objetivos para futuros sprints passa por melhorar a nossa eficácia, deixando menos critérios de aceitação por implementar, assim como implementar alguns que foram deixados do sprint anterior.

\vspace{30pt}
\section{Definition of Done}
\begin{itemize}
  \item Todas as user stories devem ser realizadas, sendo que alguns critérios de aceitação, previamente acordados em grupo, não serão realizados.
  \item Os diferentes membros da equipa devem concordaar com as user stories a serem desenvolvidas, quer no sprint planning, quer as user stories que são adicionadas durante o sprint.
  \item Todas as funcionalidades implementadas foram testadas pelos diferentes membros da equipa.
  \item Todas as funcionalidades implementadas foram apresentadas aos membros da equipa que as aprovaram.
  \item O Sprint Review foi preparado.
\end{itemize}
\end{document}


